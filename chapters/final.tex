\chapter{Conclusiones}
\label{ch:conclusiones}

\section{Objetivos cumplidos}
\label{sec:objetivos}

Durante el transcurso del trabajo, puedo considerar los siguientes objetivos como completados:

\begin{itemize}
    \item Se ha estudiado el uso de la virtualización como herramienta para asegurar servicios.
    \item Se ha estudiado el uso de oVirt como plataforma de gestión del \emph{datacenter}
    \item Assegurar els servidors de l'IES La Bastida a través de la virtualització
    \item Se ha probado la migración de máquinas virtuales, aunque con resultados variados.
    \item Estudiados los mecanismos de copias de seguridad en oVirt.
\end{itemize}

Considerando como el único objetivo que no ha sido completado satisfactoriamente de forma regular es la migración de máquinas virtuales, considero el proyecto un éxito, si menos tomándolo como una prueba de concepto de lo que se puede hacer con oVirt.

Resulta una lástima que hayan cosas que no se hayan podido completar satisfactoriamente, o averiguar por qué fallaban, como en el caso de la migración de máquinas virtuales, pero desde bastante pronto en el proyecto decidí que éste iba a ser más una prueba de concepto que un proyecto a ser implementado tal y como saliera de éste crédito.

\section{Análisis de la planificación y metodología}
\label{sec:plan_metod}

Acercándonos al final del proyecto, considero que la metodología que se decidió seguir este año, realmente no difiere en mucho a como se había hecho en años anteriores. El seguimiento del proyecto fue casi inexistente, cosa que a mí personalmente me llevó a una dinámica que no dista demasiado de cualquier otro proyecto. Por supuesto que adherise rigurosamente al plan de trabajo me correspondía a mí, pero se hubiera agradecido una mayor supervisión, sobretodo en materia de contenido del documento.

\section{Planes de futuro para el proyecto}
\label{sec:futuro}

Una vez concluido el curso, continuaré trabajando en el proyecto, y experimentando con oVirt y otras herramientas, pues el objetivo era encontrar la manera de implementar un sistema alternativo en el IES La Bastida.

Uno de los puntos que más ganas tengo de investigar es el uso de ZFS, recientemente incorporado a Ubuntu 16.04, como sistema de ficheros detrás del almacenamiento en red. En la comparativa de ZFS, btrfs, XFS, ext4 y LVM/KVM realizada por Gionatan Danti \cite{danticomparison}, ZFS sale bastante bien parada de la comparativa con los otros sistemas de ficheros, y ofrece también una serie de características adicionales, como la caché ARC, soporte nativo de \emph{snapshots}, \emph{striping} dinámico, deduplicación de datos, y más.

 Existe también otro punto importante que no he podido tratar en el proyecto por falta de tiempo, el uso de soluciones basadas en \textbf{contenedores}, como Proxmox VE. 

%% \section{Opiniones personales}
%% \label{sec:personal}

