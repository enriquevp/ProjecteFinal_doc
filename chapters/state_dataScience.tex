\section{Estado del arte: ciencia de los datos}
\label{sec:state_dataScience}

\subsection{Introducción}
\label{subsec:state_dataScience_intro}

La ciencia de los datos es algo difícil de definir. Apareció mencionado por
primera vez como datalogía, siendo un término que Peter Naur utilizó en 1966
porque lo consideraba más apropiado que ``ciencias de la computación''. Incluso
hoy en día, aunque el término lleve usándose en su forma actual más de veinte
años, no existe ninguna definición formal sobre qué entraña exactamente, sino
que simplemente diferentes investigadores y artículos han aportado su opinión en
qué quiere decir el término. Chikio Hayashi \cite{Hayashi1998} la definió como
\emph{un concepto para unificar estadística, análisis de datos y métodos
  relacionados}, y Zhu et. al \cite{Zhu2011} la definieron de forma elegante
como \emph{una nueva ciencia cuyo objeto de estudio son los datos}.


\subsection{Flujo de trabajo}
\label{subsec:state_dataScience_workflow}

\subsubsection{Obtención de los datos}
\label{subsec:state_dataScience_workflow_1}

\subsubsection{Análisis explorativo}
\label{subsec:state_dataScience_workflow_2}

\subsubsection{Modelado}
\label{subsec:state_dataScience_workflow_3}

\subsubsection{Comunicación de los resultados}
\label{subsec:state_dataScience_workflow_4}





