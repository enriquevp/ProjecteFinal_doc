\section{Introducción}
\label{sec:aim}

\subsection{Contexto y justificación}
\label{subsec:context}

Las bases de datos relacionales llevan con nosotros desde hace más de 40 años.
Están basadas en el modelo relacional (\cite{Codd:1970:RMD:362384.362685}), en el que los datos se representan en tuplas, agrupadas en
relaciones, que por lo general, representan entidades. Por tanto, almacenan
\textbf{datos estructurados}, y resultan ideales en situaciones en las cuales el
modelo de los datos está rígidamente definido.

De todas maneras, la mayor parte de la información que se genera a
diario en Internet (pista: mucha), no se ajusta a niguna estructura, y
por tanto, no puede ser almacenada tal cual en sistemas con esquemas definidos.
En respuesta a ello, y al rápido aumento en la cantidad de datos a procesar,
surgieron las \textbf{bases de datos NoSQL}. Al no usar el tradicional modelo
relacional, éstas resultan mucho más adecuadas para manejar datos sin ningún
tipo de organización o estructura. 

Otro de los conceptos al que va enfocado el proyecto es el de ciencia de los
datos, posiblemente el último ``no va más'' en el mercado. Hasta ha sido
clasificado por Glassdoor como el mejor trabajo en América a inicios de 2016
\footnote{\url{https://www.glassdoor.com/blog/25-jobs-america-2016/}}. Uno de
los motivos por los cuales decidimos explorar éste concepto en el proyecto es el
hecho de que aunque lleva ya tiempo establecido, aún sigue sin un currículo
estandarizado, o una definición concreta a la que acogerse.

\subsection{Objetivos}
\label{subsec:objectives}

Los objetivos principales del proyecto son:

\begin{itemize}
    \item Estudiar y conocer las bases de datos NoSQL, en concreto Apache Cassandra
      \begin{itemize}
      \item Identificar los usos ideales para su uso.
      \end{itemize}
    \item Entender el diseño y modelado de datos en Apache Cassandra.
    \item Utilizar la \emph{API Streaming} de Twitter para almacenar twits que
      hagan referencia a un \emph{hashtag}.
    \item Entender y familiarizarse con el flujo de trabajo del científico de datos.
      \begin{itemize}
      \item Realizar análisis de sentimiento de los twits obtenidos mediante Python.
      \end{itemize}
\end{itemize}

\subsection{Metodología}
\label{subsec:metodologia}

El proyecto se dividirá en dos partes: la primera consistirá en un estudio de
las bases de datos NoSQL, concretamente Apache Cassandra, y una
comparativa con bases de datos relacionales. Además, exploraremos el concepto de
\emph{ciencia de los datos}, una nueva profesión que combina diferentes
disciplinas, desde la programación hasta las matemáticas y la estadística,
centrada en en la solución de problemas / obtención de conocimiento a partir de
información. 

La segunda parte consistirá en utilizar lo aprendido en la primera parte para
implementar un sistema que aproveche las aptitudes de Cassandra para el
\emph{big data} y el análisis de datos. A ése fin, utilizaremos la API Streaming
de Twitter para obtener twits y almacenarlos en Apache Cassandra, sobre los
cuales luego utilizaremos Python para realizar análisis de sentimiento sobre los
twits.

\subsection{Características técnicas}
\label{subsec:planificació}

Aquí se detallan las tecnologías y herramientas que se prevé
utilizar durante el proyecto:

\begin{itemize}
    \item \textbf{Documento de la memoria: } \LaTeX
    \item \textbf{SGBD NoSQL: } Apache Cassandra
    \item \textbf{Lenguaje de programación: } Python 
    \item \textbf{Librerías: } 
      \begin{itemize}
      \item Datastax Cassandra driver for Python
      \item Tweepy
      \item Pandas
      \item Natural Language Toolkit (NLTK)
      \end{itemize}
    \item \textbf{Presentación: } Reveal.js
    \item \textbf{Aplicaciones auxiliares: } 
      \begin{itemize}
        \item Jupyter Notebook
      \end{itemize}
\end{itemize}

\textbf{Nota:} Se irá actualizando ésta lista a medida que se van descubriendo
nuevas herramientas y librerías.


\subsection{Planificación del proyecto}
\label{subsec:planificació}

\subsubsection{Lista de tareas}
\label{subsubsec:tasklist}

\begin{table}[!htb]
\centering
\begin{tabular}{l|l|l|l|l|l|}
\hline
\multicolumn{5}{|c|}{Tarea}                                                                                                       & Predecesora \\ \hline
\multicolumn{2}{|l|}{\textbf{1}} & \multicolumn{3}{l|}{\textbf{Redactar  estudio}}                                                &             \\ \hline
              & 1.1              & \multicolumn{3}{l|}{Describir brevemente las bases de datos NoSQL}                             &             \\ \cline{2-6} 
              & 1.2              & \multicolumn{3}{l|}{Estudiar Apache Cassandra: su modelo de datos, y sus casos de uso ideales} & 1.1         \\ \cline{2-6} 
              & 1.3              & \multicolumn{3}{l|}{Realizar una comparativa con las bases de datos relacionales}              & 1.2         \\ \cline{2-6} 
              & 1.4              & \multicolumn{3}{l|}{Definir el concepto de ciencia de los datos}                               &             \\ \cline{2-6} 
              & 1.5              & \multicolumn{3}{l|}{Entender la metodología del científico de datos}                           & 1.4        \\ \hline

\multicolumn{2}{|l|}{\textbf{2}} & \multicolumn{3}{l|}{\textbf{Implementación del proyecto}}                                      & \textbf{1}  \\ \hline
              & 2.1              & \multicolumn{3}{l|}{Estudio de algoritmos de clasificado y selección del mismo.}               &             \\ \cline{2-6} 
              & 2.2              & \multicolumn{3}{l|}{Diseño e implementación de la base de datos}                               & 2.1         \\ \cline{2-6} 
              & 2.3              & \multicolumn{3}{l|}{Introducción a los algoritmos de clasificado}                              & 2.2         \\ \cline{2-6} 
              & 2.4              & \multicolumn{3}{l|}{Implementación del programa para popular la base de datos}                 & 2.3         \\ \cline{2-6} 
              & 2.5              & \multicolumn{3}{l|}{Análisis de los datos}                                                     & 2.4         \\ \hline
\multicolumn{2}{|l|}{\textbf{2}} & \multicolumn{3}{l|}{\textbf{Redacción de la memoria}}                                          &             \\ \hline
\end{tabular}
\end{table}

\subsubsection{Diagrama Gantt}
\label{subsubsec:tasklist}

\vspace{0.45cm}
\hspace*{-1.2in}
\begin{ganttchart} [
  hgrid,
  vgrid,
  x unit = 4mm,
  time slot format = isodate,
  ]{2017-3-7}{2017-4-8}
  \gantttitlecalendar{year, month, day}\\

  \ganttbar{Introducción}{2017-3-7}{2017-3-15}\\
  \ganttlinkedbar{Apache Cassandra}{2017-3-15}{2017-4-2}\\
  \ganttlinkedbar{Comparativa}{2017-4-2}{2017-4-4}\\ 
  \ganttlink{elem0}{elem1}
  \ganttlink{elem1}{elem2}

  \ganttgroup{Ciencia de los datos}{2017-3-7}{2017-4-7}\\
  \ganttbar[name=dataIntro]{Introducción}{2017-3-7}{2017-3-15}\\
  \ganttbar[name=flujo]{Flujo de trabajo}{2017-3-15}{2017-3-22}\\
  \ganttbar[name=class]{Algoritmos de clasificado}{2017-3-22}{2017-4-7}\\

  \ganttlink{dataIntro}{flujo}
  \ganttlink{flujo}{class}
  
  \ganttmilestone{Entrega estudio}{2017-4-7}
\end{ganttchart}

\vspace{1cm}
\hspace*{-1.3in}
\begin{tikzpicture}[y = 5mm]
\begin{ganttchart} [
  hgrid,
  vgrid,
  x unit = 4mm,
  time slot format = isodate,
  bar top shift=-0.1]{2017-4-8}{2017-5-12}
  \gantttitlecalendar{year, month, day}\\

  \ganttbar[name=dbDesign]{Diseño base de datos}{2017-4-8}{2017-4-12}\\

  \ganttgroup[name=impl]{Implementación}{2017-4-12}{2017-4-25}\\
  \ganttbar[name=algo]{Selección de algoritmo}{2017-4-12}{2017-4-16}\\
  \ganttbar[name=impl2]{Implementación prog.}{2017-4-16}{2017-4-25}\\
  \ganttbar[name=unitTest]{Pruebas unitarias}{2017-4-18}{2017-4-25}\\

  \ganttlinkedbar[name=analisis]{Análisis de los datos}{2017-4-27}{2017-5-11}\\

  \ganttlink{dbDesign}{impl}
  \ganttlink{algo}{impl2}
  \ganttlink{impl}{analisis}

  \ganttmilestone{Entrega implementación}{2017-5-12}
\end{ganttchart}
\end{tikzpicture}

\begin{center}
  \textbf{Nota: } El análisis de datos es un proceso largamente iterativo, la
  longitud en el diagrama es orientativa.
\end{center}

\vspace{1cm}
\hspace*{-1.3in}
\begin{tikzpicture}[y = 5mm]
\begin{ganttchart} [
  hgrid,
  vgrid,
  x unit = 4mm,
  time slot format = isodate,
  bar top shift=-0.1]{2017-5-12}{2017-6-3}
  \gantttitlecalendar{year, month, day}\\
  \ganttbar[name=code]{Documentación del código}{2017-5-12}{2017-6-1}\\

  \ganttmilestone{Entrega del producto}{2017-6-3}
\end{ganttchart}
\end{tikzpicture}

\subsection{Descripción de los capítulos}
\label{subsec:chaps}

\begin{enumerate}
  \item \textbf{Estado del arte: }
    \begin{enumerate}
      \item \textbf{NoSQL y Apache Cassandra: } En esta sección estudiaremos
        brevemente las bases de datos NoSQL, en concreto Apache Cassandra.
        Trataremos de entenderlas, e identificar aquellas situaciones en las que
        es ideal utilizarlas.
      \item \textbf{Ciencia de los datos: } Aquí introduciremos y exploraremos el
        concepto de ciencia de los datos. Intentaremos desentrañar a qué se refiere
        exactamente, con qué herramientas se trabaja, y con qué objetivos.
        Haremos énfasis en el flujo de trabajo del científico de los datos, es
        decir, la metodología estándar en un proyecto de éstas características.
    \end{enumerate}

  \item \textbf{Implementación: } En ésta sección documentaremos
    la implementación de la base de datos, y el análisis de los twits, razonando
    cada decisión técnica tomada a lo largo del proceso.
  \item \textbf{Conclusiones: } A ser redactada en la última entrega de la
    memoria, en ésta sección daremos nuestras opiniones sobre nuestro
    aprendizaje de la temática del proyecto, y el proceso de implementación del mismo.
  \item \textbf{Anexo: } En éste capítulo se incluirán guías y descripciones
    varias como la preparación del entorno de desarrollo, y otras.
\end{enumerate}

\subsection{Formato de éste documento}
\label{subsec:format}

En el documento, se utilizarán las siguientes convenciones de formato:

    \begin{TMterminal}{}{}{Bloque de comandos}
      comandos a ser introducidos por el usuario en un intérprete de comandos
    \end{TMterminal}

    \vspace{0.2cm}

    \begin{TMcode}{Python}{}{Bloque de código}
      for i in range(1,10):
          print("Código fuente")
    \end{TMcode}

    \begin{TMbulletin}{normal}{Aviso}
      Información a tener en cuenta.
    \end{TMbulletin}

    \begin{TMbulletin}{warning}{$¡$Aviso importante!}
      Información importante a tener en cuenta.
    \end{TMbulletin}

    \begin{TMbulletin}{critical}{$¡$Aviso muy importante!}
      Información muy importante a tener en cuenta.
    \end{TMbulletin}
\clearpage
