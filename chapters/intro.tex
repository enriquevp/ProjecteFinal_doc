\section{Introducción}
\label{sec:aim}

\subsection{Contexto y justificación}
\label{subsec:context}

Las bases de datos relacionales llevan con nostros desde hace más de 40 años.
Están basadas en el modelo relacional (\cite{Codd:1970:RMD:362384.362685}), en el que los datos se representan en tuplas, agrupadas en
relaciones, que por lo general, representan entidades. Por tanto, almacenan
\textbf{datos estructurados}, y resultan ideales en situaciones en las cuales el
modelo de los datos está rígidamente definido.

De todas maneras, la mayor parte de la información que se genera a
diario en Internet (pista: mucha), no se conforma a niguna estructura, y
por tanto, no puede ser almacenada tal cual en sistemas con esquemas definidos.
En respuesta a ello, y al rápido aumento en la cantidad de datos a procesar,
surgieron las \textbf{bases de datos NoSQL}. Al no usar el tradicional modelo
relacional, éstas resultan mucho más adecuadas para manejar datos sin ningún
tipo de organización / estructura. 

Otro de los conceptos al que va enfocado el proyecto es el de ciencia de los
datos, posiblemente el último ``no va más'' en el mercado. Hasta ha sido
clasificado por Glassdoor como el mejor trabajo en América a inicios de 2016
\footnote{\url{https://www.glassdoor.com/blog/25-jobs-america-2016/}}. Todavía
no existe un currículo / definición consensuada, pero por ahora, y por no
\emph{spoilear} el documento, podemos hablar de ello como ``aquéllo que se hace
para obtener conocimiento y información de datos.''


\subsection{Objetivos}
\label{subsec:objectives}

Los objetivos principales del proyecto son:

\begin{itemize}
    \item Estudiar y conocer el SGBD NoSQL Apache Cassandra.
      \begin{itemize}
      \item Identificar los usos ideales para Apache Cassandra.
      \end{itemize}
    \item Entender el modelo de datos y consultas de Apache Cassandra.
    \item Utilizar la \emph{API Streaming} de Twitter para abocar twits
      referentes a un hashtag.
    \item Entender y familiarizarse con el flujo de trabajo del científico de datos.
      \begin{itemize}
      \item Analizar mediante Python los datos obtenidos (análisis de
        sentimiento) de los twits obtenidos.
      \end{itemize}
\end{itemize}

\subsection{Metodología}
\label{subsec:metodologia}

El proyecto se dividirá en dos partes bien diferenciadas: la primera consistirá
de un estudio de las bases de datos NoSQL, en concreto Apache Cassandra, y una
comparativa en base al modelo relacional. Además, se explorará el concepto de
\emph{ciencia de los datos}, una nueva profesión que combina diferentes
disciplinas, desde la programación hasta las matemáticas y la estadística,
centrada en en la solución de problemas / obtención de conocimiento a partir de información.

La segunda parte consistirá en utilizar lo aprendido en la primera parte para
implementar un sistema que aproveche las aptitudes de Cassandra para el
\emph{big data} y el análisis de datos. Para ello, utilizaremos la API Streaming
de Twitter para obtener twits y almacenarlos en Apache Cassandra, sobre los
cuales luego utilizaremos Python para realizar un pequeño análisis de sentimiento.

\subsection{Características técnicas}
\label{subsec:planificació}

Aquí se detallan las tecnologías y herramientas que se prevé
utilizar durante el proyecto:

\begin{itemize}
    \item \textbf{Documento de la memoria: } \LaTeX
    \item \textbf{SGBD NoSQL: } Apache Cassandra
    \item \textbf{Lenguaje de programación: } Python 
      \vspace{0.5cm}
    \item \textbf{Librerías: } 
      \begin{itemize}
      \item Datastax Cassandra driver for Python
      \item Tweepy
      \item Pandas
      \item Natural Language Toolkit (NLTK)
      \item matplotlib
      \item Jupyter Notebook
      \end{itemize}
    \item \textbf{Presentación: } Reveal.js
\end{itemize}

\subsection{Planificación del proyecto}
\label{subsec:planificació}


\subsubsection{Lista de tareas}
\label{subsubsec:tasklist}

\begin{table}[!htb]
\centering
\begin{tabular}{l|l|l|l|l|l|}
\hline
\multicolumn{5}{|c|}{Tarea}                                                                                                       & Predecesora \\ \hline
\multicolumn{2}{|l|}{\textbf{1}} & \multicolumn{3}{l|}{\textbf{Redactar  estudio}}                                                &             \\ \hline
              & 1.1              & \multicolumn{3}{l|}{Describir brevemente las bases de datos NoSQL}                             &             \\ \cline{2-6} 
              & 1.2              & \multicolumn{3}{l|}{Estudiar Apache Cassandra: su modelo de datos, y sus casos de uso ideales} & 1.1         \\ \cline{2-6} 
              & 1.3              & \multicolumn{3}{l|}{Realizar una comparativa con las bases de datos relacionales}              & 1.2         \\ \cline{2-6} 
              & 1.4              & \multicolumn{3}{l|}{Definir el concepto de ciencia de los datos}                               &             \\ \cline{2-6} 
              & 1.5              & \multicolumn{3}{l|}{Entender la metodologá del científico de datos}                            & 1.4         \\ \hline
\multicolumn{2}{|l|}{\textbf{2}} & \multicolumn{3}{l|}{\textbf{Implementación del proyecto}}                                      & \textbf{1}  \\ \hline
              & 2.1              & \multicolumn{3}{l|}{Definición del proyecto}                                                   &             \\ \cline{2-6} 
              & 2.2              & \multicolumn{3}{l|}{Diseño e implementación de la base de datos}                               & 2.1         \\ \cline{2-6} 
              & 2.3              & \multicolumn{3}{l|}{Implementación del programa para popular la base de datos}                 & 2.2         \\ \cline{2-6} 
              & 2.4              & \multicolumn{3}{l|}{Análisis de los datos}                                                     & 2.3         \\ \cline{2-6} 
\end{tabular}
\end{table}

\subsubsection{Diagrama Gantt}
\label{subsubsec:tasklist}

\vspace{1cm}
\hspace*{-1.2in}
\begin{ganttchart} [
  hgrid,
  vgrid,
  x unit = 4mm,
  time slot format = isodate,
  ]{2017-3-7}{2017-4-8}
  \gantttitlecalendar{year, month, day}\\

  \ganttbar{Introducción}{2017-3-7}{2017-3-15}\\
  \ganttlinkedbar{Apache Cassandra}{2017-3-15}{2017-4-2}\\
  \ganttlinkedbar{Comparativa}{2017-4-2}{2017-4-4}\\ 
  \ganttlink{elem0}{elem1}
  \ganttlink{elem1}{elem2}

  \ganttgroup{Ciencia de los datos}{2017-3-7}{2017-4-7}\\
  \ganttbar[name=dataIntro]{Introducción}{2017-3-7}{2017-3-15}\\
  \ganttbar[name=flujo]{Flujo de trabajo}{2017-3-15}{2017-4-5}\\
  \ganttlink{dataIntro}{flujo}
  

  \ganttmilestone{Entrega estudio}{2017-4-7}
\end{ganttchart}

\vspace{1cm}
\hspace*{-1.3in}
\begin{tikzpicture}[y = 5mm]
\begin{ganttchart} [
  hgrid,
  vgrid,
  x unit = 4mm,
  time slot format = isodate,
  bar top shift=-0.1]{2017-4-8}{2017-5-12}
  \gantttitlecalendar{year, month, day}\\

  \ganttbar[name=dbDesign]{Diseño base de datos}{2017-4-8}{2017-4-12}\\

  \ganttgroup[name=api]{API Twitter}{2017-4-12}{2017-4-25}\\
  \ganttbar[name=twitterIntro]{Introducción}{2017-4-12}{2017-4-16}\\
  \ganttbar[name=impl]{Implementación prog.}{2017-4-16}{2017-4-25}\\

  \ganttlinkedbar[name=analisis]{Análisis de los datos}{2017-4-26}{2017-5-11}\\

  \ganttlink{dbDesign}{api}
  \ganttlink{twitterIntro}{impl}
  \ganttlink{api}{analisis}

  \ganttmilestone{Entrega implementación}{2017-5-12}

\end{ganttchart}

\end{tikzpicture}

\begin{center}
  \textbf{Nota: } El análisis de datos es un proceso largamente iterativo, la
  longitud en el diagrama es orientativa.
\end{center}


\subsubsection{Planificación horaria}
\label{subsubsec:hores}

\subsection{Descripción de los capítulos}
\label{subsec:chaps}

\begin{enumerate}
  \item \textbf{Estado del arte: }
    \begin{enumerate}
      \item \textbf{NoSQL y Apache Cassandra: } En esta sección estudiaremos
        brevemente las bases de datos NoSQL, en concreto Apache Cassandra.
        Trataremos de entenderlas, y identificar aquellas situaciones en las que
        resultan ideales.
        uso resulta
      \item \textbf{Ciencia de los datos: } Aquí introduciremos y exploraremos el
        concepto de ciencia de los datos. Intentaremos desentrañar a qué se refiere
        exactamente, con qué herramientas se trabaja, y con qué objetivos.
        Haremos énfasis en el flujo de trabajo del científico de los datos, es
        decir, la metodología estándar en cualquier proyecto de éstas características.
    \end{enumerate}

  \item \textbf{Implementación: }  Ésta parte también la dividiremos en las
    mismas partes que el estado del arte. Como su nombre indica, aquí
    explicaremos aquél codigo que escribamos para tanto implementar la base de
    datos, crear el \emph{dataset} con el que trabajaremos, y el posterior
    análisis del mismo.
  \item \textbf{Conclusiones: } 
  \item \textbf{Anexo: } En éste capítulo se incluirán guías y descripciones
    varia como la preparación del entorno de desarrollo, y demás.
\end{enumerate}

\subsection{Formato de éste documento}
\label{subsec:format}

En el documento, se utilizarán las siguientes convenciones de formato:

    \begin{TMterminal}{}{}{Bloque de comandos}
      comandos a ser introducidos por el usuario en un intérprete de comandos
    \end{TMterminal}

    \begin{TMcode}{Python}{}{Bloque de código}
      for i in range(1,10):
          print("Código fuente")
    \end{TMcode}
\clearpage
