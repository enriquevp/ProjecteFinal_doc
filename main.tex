\documentclass[12pt,a4paper,oneside,fleqn]{tmarticle}
\usepackage[]{lipsum}
\usepackage[spanish]{babel}
\usepackage[utf8]{inputenc}
\usepackage[scaled=0.85]{beramono}
\usepackage[T1]{fontenc}
\usepackage{hyperref}
\usepackage{url}
\usepackage{ caption}
\usepackage{subcaption}
\usepackage[backend=biber,maxcitenames=3,style=authoryear,sorting=none]{biblatex}
\renewcommand*{\nameyeardelim}{\addcomma\space}
\usepackage[toc,section=section]{glossaries}
\makeglossaries{}

% GLOSSARY ENTRIES
\newglossaryentry{hashtag} {
  name=hashtag,
  description={texto precedido por una almohadilla, representando una
    etiqueta que ayuda a encontrar mensajes que hablen del mismo tema}
}

\newglossaryentry{variableCategorica} {
  name=variable categórica,
  description={se dice de una variable que puede tomar un valor de un juego
    limitado de valores}
}

\newglossaryentry{variableContinua} {
  name=variable contínua,
  description={se dice de una variable numérica o de fecha, que tiene infinitos
    valores entre dos valores determinados}
}

\newglossaryentry{variableDiscreta} {
  name=variable discreta,
  description={se dice de una variable numérica que tiene un número contable de
    valores entre dos valores determinadods}
}

\newglossaryentry{escaladoVertical} {
  name=escalado vertical,
  description={se dice del mecanismo de escalado en el que el rendimiento de un
    sistema se aumenta añadiéndole más recursos de hardware a una máquina existente.}
}

\newglossaryentry{escaladoHorizontal} {
  name=escalado horizontal,
  description={se dice del mecanismo de escalado en el que el rendimiento de un
    sistema se aumenta añadiendo máquinas adicionales.}
}

\usepackage{bibentry}
\addbibresource{bibliography.bib}
\usepackage{helvet}
\usepackage[]{paralist}
\renewcommand{\familydefault}{\sfdefault}
\usepackage{titlesec}
\usepackage[]{rotating}
\usepackage[]{fancyhdr}
\usepackage[a4paper,width=150mm,top=25mm,bottom=25mm,bindingoffset=6mm]{geometry}
\usepackage[]{xcolor,colortbl}
\graphicspath{{img/}{../img/}}
\usepackage{pdfpages}
\usepackage[]{svg}
\usepackage[]{amsmath}
\usepackage{pgfgantt}
\usepackage[]{amssymb}
\DeclareTextCommand{\nobreakspace}{T1}{\leavevmode\nobreak\ }
\titleclass{\subsubsubsection}{straight}[\subsection]

\newcounter{subsubsubsection}[subsubsection]
\renewcommand\thesubsubsubsection{\thesubsubsection.\arabic{subsubsubsection}}
\renewcommand\theparagraph{\thesubsubsubsection.\arabic{paragraph}} % optional; useful if paragraphs are to be numbered

% Formato de las listas enumeradas
\renewcommand{\labelenumii}{\theenumii}
\renewcommand{\theenumii}{\theenumi.\arabic{enumii}.}


\titleformat{\subsubsubsection}
  {\normalfont\normalsize\bfseries}{\thesubsubsubsection}{1em}{}
\titlespacing*{\subsubsubsection}
{0pt}{3.25ex plus 1ex minus .2ex}{1.5ex plus .2ex}

\makeatletter
\renewcommand\paragraph{\@startsection{paragraph}{5}{\z@}%
  {3.25ex \@plus1ex \@minus.2ex}%
  {-1em}%
  {\normalfont\normalsize\bfseries}}
\renewcommand\subparagraph{\@startsection{subparagraph}{6}{\parindent}%
  {3.25ex \@plus1ex \@minus.2ex}%
  {-1em}%
  {\normalfont\normalsize\bfseries}}
\def\toclevel@subsubsubsection{4}
\def\toclevel@paragraph{5}
\def\toclevel@paragraph{6}
\def\l@subsubsubsection{\@dottedtocline{4}{7em}{4em}}
\def\l@paragraph{\@dottedtocline{5}{10em}{5em}}
\def\l@subparagraph{\@dottedtocline{6}{14em}{6em}}
\makeatother
\setcounter{secnumdepth}{4}
\setcounter{tocdepth}{4}
\newtheorem*{capTheorem}{Teorema CAP}
\begin{document}
\pagestyle{fancy}
\begin{titlepage}
	\centering
        \includegraphics[scale=0.75]{puiglogo.png}\par\vspace{1,5cm}
            {\huge\bfseries Análisis de datos con Apache Cassandra y Python\par}
	\vspace{0.5cm}
	    {\scshape\Large Memoria\par}
	\vspace{0.5cm}
	    {\scshape\Large CFGS Desarrollo de Aplicaciones Multiplataforma\par}
	\vspace{6.0cm}
  \hfill{\Large\textbf{Enrique Villarreal}}\par
  \vspace{0.1cm}
  \hfill{\Large\textbf{Adrián Asensio}}\par  
        \vspace{1.5cm}
            \hfill{\large Curso \textbf{2016-2017}\par}
            \hfill{\large\today\par}
         \vspace{2cm}
         \begin{flushright}

        \includegraphics[scale=0.75]{creativecommons.png}\par\vspace{0.05cm}

           Esta obra está sujeta a una licencia de Reconocimiento-NoComercial-SinObraDerivada 3.0 España de Creative Commons
         \end{flushright}
        \vfill
\end{titlepage}
\thispagestyle{empty}
\textbf{Resumen}
\vspace{0.5cm}

Hoy en día, dos de los conceptos más populares en la industria son el \emph{big data} y la ciencia de los datos. En éste proyecto, nos mojaremos los
pies un poco en ambos conceptos, apoyándonos del SGBD NoSQL Apache Cassandra y
Python para realizar un breve análisis de sentimiento de twits recogidos
mediante la API de Twitter.

\vspace{0.5cm}
\textbf{Palabras clave: } big data, data science, nosql, cassandra, python, twitter

\clearpage
\textbf{Abstract}
\vspace{0.5cm}

These days, two of the most popular buzzwords in IT are big data, and data
science. In this project, we will get our feet wet in both these concepts, using
the NoSQL DBMS Apache Cassandra and Python to perform sentiment analysis on
tweets collected by using the Twitter API.

\vspace{0.5cm}
\textbf{Keywords: } big data, data science, nosql, cassandra, python, twitter

\clearpage
\thispagestyle{empty}
\tableofcontents
\clearpage
\thispagestyle{empty}
\listoffigures
\clearpage
\listoftables
\clearpage
\pagenumbering{arabic}

\section{Introducción}
\label{sec:aim}

\subsection{Contexto y justificación}
\label{subsec:context}

Las bases de datos relacionales llevan con nosotros desde hace más de 40 años.
Están basadas en el modelo relacional (\cite{Codd:1970:RMD:362384.362685}), en el que los datos se representan en tuplas, agrupadas en
relaciones, que por lo general, representan entidades. Por tanto, almacenan
\textbf{datos estructurados}, y resultan ideales en situaciones en las cuales el
modelo de los datos está rígidamente definido.

De todas maneras, la mayor parte de la información que se genera a
diario en Internet (pista: mucha), no se ajusta a niguna estructura, y
por tanto, no puede ser almacenada tal cual en sistemas con esquemas definidos.
En respuesta a ello, y al rápido aumento en la cantidad de datos a procesar,
surgieron las \textbf{bases de datos NoSQL}. Al no usar el tradicional modelo
relacional, éstas resultan mucho más adecuadas para manejar datos sin ningún
tipo de organización o estructura. 

Otro de los conceptos al que va enfocado el proyecto es el de ciencia de los
datos, posiblemente el último ``no va más'' en el mercado. Hasta ha sido
clasificado por Glassdoor como el mejor trabajo en América a inicios de 2016
\footnote{\url{https://www.glassdoor.com/blog/25-jobs-america-2016/}}. Uno de
los motivos por los cuales decidimos explorar éste concepto en el proyecto es el
hecho de que aunque lleva ya tiempo establecido, aún sigue sin un currículo
estandarizado, o una definición concreta a la que acogerse.

\subsection{Objetivos}
\label{subsec:objectives}

Los objetivos principales del proyecto son:

\begin{itemize}
    \item Estudiar y conocer las bases de datos NoSQL, en concreto Apache Cassandra
      \begin{itemize}
      \item Identificar los usos ideales para su uso.
      \end{itemize}
    \item Entender el diseño y modelado de datos en Apache Cassandra.
    \item Utilizar la \emph{API Streaming} de Twitter para almacenar twits que
      hagan referencia a un \emph{hashtag}.
    \item Entender y familiarizarse con el flujo de trabajo del científico de datos.
      \begin{itemize}
      \item Realizar análisis de sentimiento de los twits obtenidos mediante Python.
      \end{itemize}
\end{itemize}

\subsection{Metodología}
\label{subsec:metodologia}

El proyecto se dividirá en dos partes: la primera consistirá en un estudio de
las bases de datos NoSQL, concretamente Apache Cassandra, y una
comparativa con bases de datos relacionales. Además, exploraremos el concepto de
\emph{ciencia de los datos}, una nueva profesión que combina diferentes
disciplinas, desde la programación hasta las matemáticas y la estadística,
centrada en en la solución de problemas / obtención de conocimiento a partir de
información. 

La segunda parte consistirá en utilizar lo aprendido en la primera parte para
implementar un sistema que aproveche las aptitudes de Cassandra para el
\emph{big data} y el análisis de datos. A ése fin, utilizaremos la API Streaming
de Twitter para obtener twits y almacenarlos en Apache Cassandra, sobre los
cuales luego utilizaremos Python para realizar análisis de sentimiento sobre los
twits.

\subsection{Características técnicas}
\label{subsec:planificació}

Aquí se detallan las tecnologías y herramientas que se prevé
utilizar durante el proyecto:

\begin{itemize}
    \item \textbf{Documento de la memoria: } \LaTeX
    \item \textbf{SGBD NoSQL: } Apache Cassandra
    \item \textbf{Lenguaje de programación: } Python 
    \item \textbf{Librerías: } 
      \begin{itemize}
      \item Datastax Cassandra driver for Python
      \item Tweepy
      \item Pandas
      \item Natural Language Toolkit (NLTK)
      \end{itemize}
    \item \textbf{Presentación: } Reveal.js
    \item \textbf{Aplicaciones auxiliares: } 
      \begin{itemize}
        \item Jupyter Notebook
      \end{itemize}
\end{itemize}

\textbf{Nota:} Se irá actualizando ésta lista a medida que se van descubriendo
nuevas herramientas y librerías.


\subsection{Planificación del proyecto}
\label{subsec:planificació}

\subsubsection{Lista de tareas}
\label{subsubsec:tasklist}

\begin{table}[!htb]
\centering
\begin{tabular}{l|l|l|l|l|l|}
\hline
\multicolumn{5}{|c|}{Tarea}                                                                                                       & Predecesora \\ \hline
\multicolumn{2}{|l|}{\textbf{1}} & \multicolumn{3}{l|}{\textbf{Redactar  estudio}}                                                &             \\ \hline
              & 1.1              & \multicolumn{3}{l|}{Describir brevemente las bases de datos NoSQL}                             &             \\ \cline{2-6} 
              & 1.2              & \multicolumn{3}{l|}{Estudiar Apache Cassandra: su modelo de datos, y sus casos de uso ideales} & 1.1         \\ \cline{2-6} 
              & 1.3              & \multicolumn{3}{l|}{Realizar una comparativa con las bases de datos relacionales}              & 1.2         \\ \cline{2-6} 
              & 1.4              & \multicolumn{3}{l|}{Definir el concepto de ciencia de los datos}                               &             \\ \cline{2-6} 
              & 1.5              & \multicolumn{3}{l|}{Entender la metodología del científico de datos}                           & 1.4        \\ \hline

\multicolumn{2}{|l|}{\textbf{2}} & \multicolumn{3}{l|}{\textbf{Implementación del proyecto}}                                      & \textbf{1}  \\ \hline
              & 2.1              & \multicolumn{3}{l|}{Estudio de algoritmos de clasificado y selección del mismo.}               &             \\ \cline{2-6} 
              & 2.2              & \multicolumn{3}{l|}{Diseño e implementación de la base de datos}                               & 2.1         \\ \cline{2-6} 
              & 2.3              & \multicolumn{3}{l|}{Introducción a los algoritmos de clasificado}                              & 2.2         \\ \cline{2-6} 
              & 2.4              & \multicolumn{3}{l|}{Implementación del programa para popular la base de datos}                 & 2.3         \\ \cline{2-6} 
              & 2.5              & \multicolumn{3}{l|}{Análisis de los datos}                                                     & 2.4         \\ \hline
\multicolumn{2}{|l|}{\textbf{2}} & \multicolumn{3}{l|}{\textbf{Redacción de la memoria}}                                          &             \\ \hline
\end{tabular}
\end{table}

\subsubsection{Diagrama Gantt}
\label{subsubsec:tasklist}

\vspace{0.45cm}
\hspace*{-1.2in}
\begin{ganttchart} [
  hgrid,
  vgrid,
  x unit = 4mm,
  time slot format = isodate,
  ]{2017-3-7}{2017-4-8}
  \gantttitlecalendar{year, month, day}\\

  \ganttbar{Introducción}{2017-3-7}{2017-3-15}\\
  \ganttlinkedbar{Apache Cassandra}{2017-3-15}{2017-4-2}\\
  \ganttlinkedbar{Comparativa}{2017-4-2}{2017-4-4}\\ 
  \ganttlink{elem0}{elem1}
  \ganttlink{elem1}{elem2}

  \ganttgroup{Ciencia de los datos}{2017-3-7}{2017-4-7}\\
  \ganttbar[name=dataIntro]{Introducción}{2017-3-7}{2017-3-15}\\
  \ganttbar[name=flujo]{Flujo de trabajo}{2017-3-15}{2017-3-22}\\
  \ganttbar[name=class]{Algoritmos de clasificado}{2017-3-22}{2017-4-7}\\

  \ganttlink{dataIntro}{flujo}
  \ganttlink{flujo}{class}
  
  \ganttmilestone{Entrega estudio}{2017-4-7}
\end{ganttchart}

\vspace{1cm}
\hspace*{-1.3in}
\begin{tikzpicture}[y = 5mm]
\begin{ganttchart} [
  hgrid,
  vgrid,
  x unit = 4mm,
  time slot format = isodate,
  bar top shift=-0.1]{2017-4-8}{2017-5-12}
  \gantttitlecalendar{year, month, day}\\

  \ganttbar[name=dbDesign]{Diseño base de datos}{2017-4-8}{2017-4-12}\\

  \ganttgroup[name=impl]{Implementación}{2017-4-12}{2017-4-25}\\
  \ganttbar[name=algo]{Selección de algoritmo}{2017-4-12}{2017-4-16}\\
  \ganttbar[name=impl2]{Implementación prog.}{2017-4-16}{2017-4-25}\\
  \ganttbar[name=unitTest]{Pruebas unitarias}{2017-4-18}{2017-4-25}\\

  \ganttlinkedbar[name=analisis]{Análisis de los datos}{2017-4-27}{2017-5-11}\\

  \ganttlink{dbDesign}{impl}
  \ganttlink{algo}{impl2}
  \ganttlink{impl}{analisis}

  \ganttmilestone{Entrega implementación}{2017-5-12}
\end{ganttchart}
\end{tikzpicture}

\begin{center}
  \textbf{Nota: } El análisis de datos es un proceso largamente iterativo, la
  longitud en el diagrama es orientativa.
\end{center}

\vspace{1cm}
\hspace*{-1.3in}
\begin{tikzpicture}[y = 5mm]
\begin{ganttchart} [
  hgrid,
  vgrid,
  x unit = 4mm,
  time slot format = isodate,
  bar top shift=-0.1]{2017-5-12}{2017-6-3}
  \gantttitlecalendar{year, month, day}\\
  \ganttbar[name=code]{Documentación del código}{2017-5-12}{2017-6-1}\\

  \ganttmilestone{Entrega del producto}{2017-6-3}
\end{ganttchart}
\end{tikzpicture}

\subsection{Descripción de los capítulos}
\label{subsec:chaps}

\begin{enumerate}
  \item \textbf{Estado del arte: }
    \begin{enumerate}
      \item \textbf{NoSQL y Apache Cassandra: } En esta sección estudiaremos
        brevemente las bases de datos NoSQL, en concreto Apache Cassandra.
        Trataremos de entenderlas, e identificar aquellas situaciones en las que
        es ideal utilizarlas.
      \item \textbf{Ciencia de los datos: } Aquí introduciremos y exploraremos el
        concepto de ciencia de los datos. Intentaremos desentrañar a qué se refiere
        exactamente, con qué herramientas se trabaja, y con qué objetivos.
        Haremos énfasis en el flujo de trabajo del científico de los datos, es
        decir, la metodología estándar en un proyecto de éstas características.
    \end{enumerate}

  \item \textbf{Implementación: } En ésta sección documentaremos
    la implementación de la base de datos, y el análisis de los twits, razonando
    cada decisión técnica tomada a lo largo del proceso.
  \item \textbf{Conclusiones: } A ser redactada en la última entrega de la
    memoria, en ésta sección daremos nuestras opiniones sobre nuestro
    aprendizaje de la temática del proyecto, y el proceso de implementación del mismo.
  \item \textbf{Anexo: } En éste capítulo se incluirán guías y descripciones
    varias como la preparación del entorno de desarrollo, y otras.
\end{enumerate}

\subsection{Formato de éste documento}
\label{subsec:format}

En el documento, se utilizarán las siguientes convenciones de formato:

    \begin{TMterminal}{}{}{Bloque de comandos}
      comandos a ser introducidos por el usuario en un intérprete de comandos
    \end{TMterminal}

    \vspace{0.2cm}

    \begin{TMcode}{Python}{}{Bloque de código}
      for i in range(1,10):
          print("Código fuente")
    \end{TMcode}

    \begin{TMbulletin}{normal}{Aviso}
      Información a tener en cuenta.
    \end{TMbulletin}

    \begin{TMbulletin}{warning}{$¡$Aviso importante!}
      Información importante a tener en cuenta.
    \end{TMbulletin}

    \begin{TMbulletin}{critical}{$¡$Aviso muy importante!}
      Información muy importante a tener en cuenta.
    \end{TMbulletin}
\clearpage

\section{Estado del arte: bases de datos NoSQL}
\label{sec:state_dataScience}

\subsection{Introducción}
\label{subsec:state_dataScience_intro}

Un sistema gestor de bases de datos NoSQL es aquél que prescinde del modelo
relacional a la hora de almacenar la información, y por ende del SQL tradicional
como lenguaje de consultas. Éstas surgen debido a algunas carencias o problemas
de las bases de datos relacionales, principalmente la \textbf{escalabilidad
  horizontal}, es decir, la adición de nuevos nodos para reducir el
\emph{downtime} del sistema en general.

\subsection{Tipos de bases de datos NoSQL}
\label{subsec:state_NoSQL_types}


Existen varios tipos de bases de datos NoSQL, generalmente clasificadas por el
modelo de datos que emplean. Éstas clasificaciones no deberían ser tomadas
literalmente, ya que muchas de las bases de datos que mencionaremos como
ejemplos, implementan varios modelos de datos al mismo tiempo.

\paragraph{Teorema de CAP}
\mbox{}\\

En los sistemas distribuidos, el \textbf{teorema de CAP}, formulado por Eric
Brewer, dice lo siguiente:

\begin{capTheorem}
  Es imposible para un sistema distribuido ofrecer más de dos de las siguientes
  garantías al mismo tiempo: consistencia, disponibilidad y tolerancia de particiones.
\end{capTheorem}

En el contexto del teorema, cada una de las garantías se define tal que:

\begin{itemize}
\item \textbf{Consistencia: } Cada lectura recibe como respuesta la escritura
  más reciente, o un error.
\item \textbf{Disponibilidad: } Cada petición recibe una respuesta no-errónea,
  sin garantizar que contenga la escritura más reciente.
\item \textbf{Tolerancia de particiones: } El sistema continúa funcionando
  aunque se degrade el funcionamiento de la red entre los nodos, o se pierdan
  mensajes entre ellos.
\end{itemize}

En realidad, la elección real es entre o consistencia, o disponibilidad, ya que
debido al hecho de que una caída de parte del sistema no es común, únicamente
nos vemos en la situación de escoger entre consistencia o disponibilidad, cuando
ocurre un particionado en el sistema. Mientras el sistema esté funcionando con
total normalidad, se puede garantizar tanto la disponibilidad como la
consistencia del mismo.

Si el sistema permite la lectura antes de que todos los nodos tengan la última
información escrita, el sistema será catalogado como uno consistente y
tolerante a particiones (CP), mientras que si bloquea las lecturas hasta que
todos los nodos se hayan actualizado, el sistema será disponible y tolerante a
particiones (AP).

\subsubsection{Almacenes de columnas}
\label{subsubsec:state_NoSQL_types_column}

Los almacenes de columnas, bases de datos tabulares o orientados a columnas, son
bases de datos que en lugar de serializar los datos en forma de filas como lo
haría un SGBD relacional, lo hace en columnas. Veamos un ejemplo:

Gracias a éste sistema, operaciones como consultar todos los registros que contengan
un valor específico son mucho más eficientes en éste tipo de de SGBDs. Debido a
ésto, resultan ideales en entornos OLAP (\emph{Online analytical processing},
procesado analítico en línea), en los que generalmente el volumen de lecturas
es mucho mayor al de inserciones / actualizaciones.

Ejemplos: Apache Cassandra, SAP Hana, Sybase IQ

\subsubsection{Bases de datos documental}
\label{subsubsec:state_NoSQL_types_doc}

En una base de datos documental, la información se almacena en un formato
semi-estructurado: es decir, con ficheros siguiendo un formato de documento
específico, como podrían ser XML, JSON y YAML.

Éstos tipos de documentos almacenan toda la información referente al objeto que
describen en el mismo documento, permitiendo que se añada información a
documentos específicos sin tener que mantener una estructura consistente en
todos los documentos.

En el espectro del teorema CAP, las bases de datos documentales se sitúan principalmente en
el vértice de la tolerancia de particiones. Las bases de datos documentales AP (Available,
partition-tolerant) como 

\subsubsection{Almacenes clave-valor}
\label{subsubsec:state_NoSQL_types_keyval}

Las bases de datos clave-valor almacenan la información en \emph{arrays}
asociativos, o dicho de otra manera, diccionarios. Cada diccionario contiene
registros o objetos, y de la misma manera que en las bases de datos
documentales, cada uno puede albergar diferente información.

En éstas bases de datos, el valor de una clave es completamente opaco, por lo
que es imposible filtrar por el contenido del registro. Al no disponer de
lenguaje de consultas, la interacción con la base de datos se realiza mediante
los comandos \texttt{get}, \texttt{put} y \texttt{delete}.

\subsubsection{Bases de datos orientada a grafos}
\label{subsubsec:state_NoSQL_types_graph}

Los grafos son una estructura de datos, que modelan la información según
relaciones entre objetos:

\begin{figure}[ht!]
  \centering
  \includegraphics[scale=0.75]{img/graphDB.png}
  \caption{\label{fig:grafo} Base de datos de grafos}
\end{figure}
\
Los grafos se basan en los siguientes tres conceptos:

\begin{enumerate}
\item Los \textbf{nodos} representan cualquier entidad, y se pueden considerar
  equivalentes a otros objetos como un registro en una base de datos relacional,
  documento en una documental, y así.
 \item Las \textbf{aristas}, también llamadas relaciones, son las líneas que
   conectan los nodos, representando las relaciones entre ellos.
 \item Las \textbf{propiedades} son etiquetas asociadas a los nodos
   nodo.
\end{enumerate}

En contraste a las bases de datos relacionales, éstas almacenan directamente las
relaciones entre nodos, por lo que, en teoría, las consultas que deban enlazar información
son más veloces en bases de datos de grafos, ya que un SGBD relacional debe
realizar una \emph{join} en dos tablas diferentes. Ésta eficiencia se va
acentuando conforme la complejidad de la consulta aumenta: una base de datos
orientada a grafos sólo debe seguir las aristas a partir del nodo inicial.

Algunos ejemplos de bases de datos orientadas a grafos son AllegroGraph, Neo4j y
Openlink Virtuoso.

\subsubsection{Bases de datos orientadas a objetos}
\label{subsubsec:state_NoSQL_types_object}

En éste tipo de bases de datos, la información se almacena como objetos en el
paradigma homónimo. Según el ``Object-Oriented Database Manifesto'', las bases
de datos orientadas a objetos difieren de la serialización de los mismos en que
deben proveer concurrencia, capacidad de recuperación, y algún mecanismo para
consultar sobre éstos objetos.
\section{Estado del arte: ciencia de los datos}
\label{sec:state_dataScience}

\subsection{Introducción}
\label{subsec:state_dataScience_intro}

La ciencia de los datos es algo difícil de definir. Apareció mencionado por
primera vez como datalogía, siendo un término que Peter Naur utilizó en 1966
porque lo consideraba más apropiado que ``ciencias de la computación''. Incluso
hoy en día, aunque el término lleve usándose en su forma actual más de veinte
años, no existe ninguna definición formal sobre qué entraña exactamente, sino
que simplemente diferentes investigadores y artículos han aportado su opinión en
qué quiere decir el término. Chikio Hayashi \cite{Hayashi1998} la definió como
\emph{un concepto para unificar estadística, análisis de datos y métodos
  relacionados}, y Zhu et. al \cite{Zhu2011} la definieron de forma elegante
como \emph{una nueva ciencia cuyo objeto de estudio son los datos}.


\subsection{Flujo de trabajo}
\label{subsec:state_dataScience_workflow}

\subsubsection{Obtención de los datos}
\label{subsec:state_dataScience_workflow_1}

\subsubsection{Análisis explorativo}
\label{subsec:state_dataScience_workflow_2}

\subsubsection{Modelado}
\label{subsec:state_dataScience_workflow_3}

\subsubsection{Comunicación de los resultados}
\label{subsec:state_dataScience_workflow_4}






%\section{Apache Cassandra}
\label{sec:ovirt_install}

\subsection{Requerimientos mínimos}
\label{subsec:minreqs}

Apache Cassandra necesita para funcionar:

\begin{itemize}
  \item Java 8 o superior
  \item Para usar cqlsh, Python 2.7 o superior
  \item En términos de hardware \footnote{\url{https://wiki.apache.org/cassandra/CassandraHardware}}, se recomienda:
    \begin{itemize}
      \item 4 GB de RAM mínimo en entornos virtualizados (EC2 Large) 
      \item Cpu de 8 núcleos en servidores físicos, para entornos virtualizados
        se recomienda cualquiera que soporte \emph{CPU bursting}.
      \item DISCO
    \end{itemize}
\end{itemize}



\subsection{Instalación}
\label{subsec:instalación}

\begin{enumerate}
  \item Primeramente, se deben añadir el repositorio y la clave:
    \begin{TMterminal}{}{}{Añadiendo el repositorio}
      echo "deb http://debian.datastax.com/community stable main" | sudo tee -a /etc/apt/sources.list.d/cassandra.sources.list
    \end{TMterminal}

    \begin{TMterminal}{}{}{Añadiendo la clave}
      curl -L http://debian.datastax.com/debian/repo_key | sudo apt-key add
    \end{TMterminal}

  \item Una vez añadidos, ya podemos proceder con la instalación:
    \begin{TMterminal}{}{}{Instalando Apache Cassandra }
      sudo apt-get update && sudo apt-get install cassandra 
    \end{TMterminal}

  \item Cuando ya esté instalado, sería buena idea arrancarlo y habilitarlo en
    cada reinicio del sistema:
    \begin{TMterminal}{}{}{Habilitando servicio}
      systemctl start cassandra
      systemctl enable cassandra
    \end{TMterminal}

  \item Para comprobar el estado del servidor, podemos utilizar el comando
    \texttt{nodetool}:
    \begin{TMterminal}{}{}{Comprobando el estado de Cassandra}
      nodetool status

      Datacenter: datacenter1
      =======================
      Status=Up/Down
      |/ State=Normal/Leaving/Joining/Moving
      -- Address Load Tokens Owns (effective) Host ID Rack
      UN 127.0.0.1 222.14 KB 256 100.0% 2a0b7fa9-23c6-40d3-83a4-e6c06e2f5736 rack1
      [root@cassandra ~]#

    \end{TMterminal}
    

\end{enumerate}




% \chapter{Implementando oVirt}
\label{sec:ovirt}

\section{Instalación del motor}
\label{sec:ovirt_install}

Una vez instalado el sistema operativo en el nodo de control (Fedora, RHEL, o CentOS 7)\footnote{Léase anexo para el proceso de instalación de CentOS}, hace falta actualizar los paquetes del sistema:

\begin{TMterminal}{}{}{Actualizando paquetes}
  yum -y update
\end{TMterminal}

El siguiente paso para instalar oVirt es subscribirnos a su repositorio. Instalar el siguiente RPM es similar a añadir un PPA en distribuciones basadas en Ubuntu en el hecho de que actualizaciones a éste paquete deberían aparecer al hacer más \texttt{yum update} en el futuro.

\begin{TMterminal}{}{}{Instalando oVirt}
  yum install http://plain.resources.ovirt.org/pub/yum-repo/ovirt-release35.rpm
  yum -y install ovirt-engine
\end{TMterminal}

Cuando el paquete se haya instalado, ejecutaremos el instalador del motor, que nos hará una serie de preguntas. El siguiente es un ejemplo del mismo instalador, con las respuestas por defecto envueltas entre corchetes \cite{ovirt}.

\begin{TMterminal}{}{}{Instalación del motor}
  engine-setup
[ INFO  ] Stage: Initializing
   [ INFO  ] Stage: Environment setup
           Configuration files: ['/etc/ovirt-engine-setup.conf.d/10-packaging.conf']
           Log file: /var/log/ovirt-engine/setup/ovirt-engine-setup-20140310163840.log
           Version: otopi-1.2.0_rc2 (otopi-1.2.0-0.7.rc2.fc19)
   [ INFO  ] Stage: Environment packages setup
   [ INFO  ] Stage: Programs detection
   [ INFO  ] Stage: Environment setup
   [ INFO  ] Stage: Environment customization
          
           --== PRODUCT OPTIONS ==--
           --== PACKAGES ==--
          
   [ INFO  ] Checking for product updates...
   [ INFO  ] No product updates found
      
           --== NETWORK CONFIGURATION ==--
          
           Host fully qualified DNS name of this server [server.name]: example.ovirt.org
           Setup can automatically configure the firewall on this system.
           Note: automatic configuration of the firewall may overwrite current settings.
           Do you want Setup to configure the firewall? (Yes, No) [Yes]:
   [ INFO  ] firewalld will be configured as firewall manager.
          
           --== DATABASE CONFIGURATION ==--
          
           Where is the Engine database located? (Local, Remote) [Local]: 
           Setup can configure the local postgresql server automatically for the engine to run. This may conflict with existing applications.
           Would you like Setup to automatically configure postgresql and create Engine database, or prefer to perform that manually? (Automatic, Manual) [Automatic]: 
          
           --== OVIRT ENGINE CONFIGURATION ==--
          
           Application mode (Both, Virt, Gluster) [Both]: 
           Default storage type: (NFS, FC, ISCSI, POSIXFS) [NFS]: 
           Engine admin password: 
           Confirm engine admin password: 
          
           --== PKI CONFIGURATION ==--
          
           Organization name for certificate [ovirt.org]: 
          
           --== APACHE CONFIGURATION ==--
          
           Setup can configure apache to use SSL using a certificate issued from the internal CA.
     
           Do you wish Setup to configure that, or prefer to perform that manually? (Automatic, Manual) [Automatic]: 
           Setup can configure the default page of the web server to present the application home page. This may conflict with existing applications.
           Do you wish to set the application as the default page of the web server? (Yes, No) [Yes]: 
          
           --== SYSTEM CONFIGURATION ==--
          
           Configure WebSocket Proxy on this machine? (Yes, No) [Yes]: 
           Configure an NFS share on this server to be used as an ISO Domain? (Yes, No) [Yes]: 
           Local ISO domain path [/var/lib/exports/iso-20140310143916]: 
           Local ISO domain ACL - note that the default will restrict access to example.ovirt.org only, for security reasons [example.ovirt.org(rw)]: 
           Local ISO domain name [ISO_DOMAIN]: 
          
           --== MISC CONFIGURATION ==--
     
           --== END OF CONFIGURATION ==--

        [ INFO  ] Stage: Setup validation
          
                    --== CONFIGURATION PREVIEW ==--
          
           Engine database name                    : engine
           Engine database secured connection      : False
           Engine database host                    : localhost
           Engine database user name               : engine
           Engine database host name validation    : False
           Engine database port                    : 5432
           NFS setup                               : True
           PKI organization                        : ovirt.org
           Application mode                        : both
           Firewall manager                        : firewalld
           Update Firewall                         : True
           Configure WebSocket Proxy               : True
           Host FQDN                               : example.ovirt.org
           NFS export ACL                          : 0.0.0.0/0.0.0.0(rw)
           NFS mount point                         : /var/lib/exports/iso-20140310143916
           Datacenter storage type                 : nfs
           Configure local Engine database         : True
           Set application as default page         : True
           Configure Apache SSL                    : True
           Please confirm installation settings (OK, Cancel) [OK]:   
   [ INFO  ] Stage: Transaction setup
   [ INFO  ] Stopping engine service
   [ INFO  ] Stopping websocket-proxy service
   [ INFO  ] Stage: Misc configuration
   [ INFO  ] Stage: Package installation
   [ INFO  ] Stage: Misc configuration
   [ INFO  ] Creating PostgreSQL 'engine' database
   [ INFO  ] Configuring PostgreSQL
   [ INFO  ] Creating Engine database schema
   [ INFO  ] Creating CA
   [ INFO  ] Configuring WebSocket Proxy
   [ INFO  ] Generating post install configuration file '/etc/ovirt-engine-setup.conf.d/20-setup-ovirt-post.conf'
   [ INFO  ] Stage: Transaction commit
   [ INFO  ] Stage: Closing up
          
           --== SUMMARY ==--
           `         SSH fingerprint: `<SSH_FINGERPRINT> `         Internal CA: `<CA_FINGERPRINT>
           Web access is enabled at: `             `[`http://example.ovirt.org:80/ovirt-engine`](http://example.ovirt.org:80/ovirt-engine) `             `[`https://example.ovirt.org:443/ovirt-engine`](https://example.ovirt.org:443/ovirt-engine)
           Please use the user "admin" and password specified in order to login into oVirt Engine
          
           --== END OF SUMMARY ==--
          
   [ INFO  ] Starting engine service
   [ INFO  ] Restarting httpd
   [ INFO  ] Restarting nfs services
   [ INFO  ] Generating answer file '/var/lib/ovirt-engine/setup/answers/20140310163837-setup.conf'
   [ INFO  ] Stage: Clean up
           Log file is located at /var/log/ovirt-engine/setup/ovirt-engine-setup-20140310163604.log
   [ INFO  ] Stage: Pre-termination
   [ INFO  ] Stage: Termination
   [ INFO  ] Execution of setup completed successfully
     
     **** Installation completed successfully ******
\end{TMterminal}

\bigskip

\begin{TMterminal}{}{}{Nota}
  Configuraré los \emph{exports} NFS a mano, así que omitiré la parte del instalador que configura un dominio ISO.
\end{TMterminal}

Una vez hecho ésto, ya estaremos listos para acceder a la interfaz web del motor, con la contraseña que le hemos dicho al instalador.

\section{Instalación de los Host}
\label{sec:host_inst}

A la hora de montar los hosts de virtualización, disponemos de varias opciones: podemos instalar el hipervisor propio de oVirt Node, que una vez configurado se añade automáticamente en el motor, o instalar el paquete encima de un CentOS o Fedora ya instalado. Ésta será la opción que documentaré.

Dicho ésto, para instalar el paquete, primero deberíamos asegurarnos que el sistema está actualizado:

\begin{TMterminal}{}{}{Actualizando paquetes}
  yum -y update
\end{TMterminal}

\bigskip

Y instalamos oVirt:

\begin{TMterminal}{}{}{Instalando paquetes}
  yum localinstall http://plain.resources.ovirt.org/pub/yum-repo/ovirt-release36.rpm
\end{TMterminal}

\bigskip

Cuando ya esté instalado, podemos añadirlo manualmente a oVirt, en la pestaña Hosts. Cuando cliquemos en \emph{New}, aparecerá el siguiente diálogo: \clearpage

\begin{figure}[h!]
  \centering
  \includegraphics[scale=0.55]{ovirt_hostinst.png}
  \caption{\label{fig:ovirthostinst} Añadiendo el host}
\end{figure}

Una vez añadido, oVirt descargará e instalará ciertos paquetes. Podemos ver el progreso de ésta tarea en la parte inferior de la pantalla, en las pestañas Events y Tasks.

\section{Configuración del almacenamiento}
\label{sec:almacenamiento}

Como ya mencioné cuando describía la arquitectura de oVirt, el almacenamiento viene proporcionado por un sistema en red. Las opciones incluyen soluciones como GlusterFS, iSCSI, o FCP, pero nosotros usaremos NFS.\@ Hay que tener en cuenta, que el tipo de almacenamiento que se use debe coincidir con el tipo por defecto que se seleccionó en el instalador.

Entonces, para configurar los \emph{exports} que usaremos en oVirt, instalamos el servidor, y lo configuramos acordemente \footnote{Ésta configuración sirve sólo de ejemplo: léase la documentación de NFS}:

\begin{TMterminal}{}{}{Configurando NFS}
  yum -y install nfs-utils
  vi /etc/exports
---------------------------
/srv/ovirt/export  *(rw,sync,no_subtree_check,all_squash,anonuid=36,anongid=36)
/srv/ovirt/data    *(rw,sync,no_subtree_check,all_squash,anonuid=36,anongid=36)
/srv/ovirt/iso     *(rw,sync,no_subtree_check,all_squash,anonuid=36,anongid=36)
/usr/ovirt/disks   *(rw,sync,no_subtree_check,all_squash,anonuid=36,anongid=36)
\end{TMterminal}

Una vez hecho ésto, debemos añadirlos en el siguiente orden en la interfaz de administración web:

\begin{enumerate}
    \item Export
    \item Data
    \item iso
\end{enumerate}

\subsection{Cargando las ISOs}
\label{subsec:isos}

Cuando ya tenemos el almacenamiento en red configurado, una particularidad de oVirt es que no podemos abocar directamente al directorio /srv/ovirt/iso las imágenes de instalación, sino que debemos usar la herramienta \texttt{engine-iso-uploader} para subirlas al datacenter, y puedan ser usadas para posteriormente instalar sistemas operativos en las máquinas virtuales. A continuación muestro un ejemplo de dicho comando:

\begin{TMterminal}{}{}{Subiendo ISOs}
  engine-iso-uploader -i nombre_del_dominio_en_ovirt upload imagen.iso
\end{TMterminal}

\subsection{Creando los discos}
\label{subsec:creando_discos}

La creación de los discos puede hacerse tanto al momento de crear la máquina virtual, o cuando el usuario quiera. De todas maneras el proceso es el mismo, detallado en la siguiente captura: 

\begin{figure}[ht!]
  \centering
  \includegraphics[scale=0.55]{ovirt_newdisk.png}
  \caption{\label{fig:create_disk} Creando discos}
\end{figure}
\clearpage

\subsection{Creando interfaces de red}
\label{subsec:creando_interfaces}

Antes de proceder finalmente a crear una máquina virtual, debemos crear la interfaz de red.

\begin{figure}[ht!]
  \centering
  \includegraphics[scale=0.55]{ovirt_newnic.png}
  \caption{\label{fig:create_nic} Creando interfaces de red}
\end{figure}

Acorde a la captura, podemos escoger varios tipos de interfaces de red:

\begin{itemize}
    \item \textbf{VirtIO:\@ }La opción más rápida y ligera, si el sistema virtualizado lo soporta
    \item \textbf{Dos modelos de tarjetas físicas reales: }Su uso supone más \emph{overhead}, ya que tienen que virtualizarse las tarjetas por completo.
    \item \textbf{NIC Passthrough: }Otorga de forma exclusiva una interfaz de red de la máquina anfitriona al sistema virtualizado.
\end{itemize}

\clearpage

\section{Creando máquinas virtuales}
\label{sec:creando_vms}

La creación de las máquinas virtuales es bastante simple. Se puede acceder al diálogo de creación de máquinas virtuales mediante el menú Virtual Machines > New. Las siguientes capturas muestran algunas de las opciones de las que disponemos a la hora de crearlas:

\begin{figure}[ht!]
  \centering
  \includegraphics[scale=0.55]{ovirt_newvm.png}
  \caption{\label{fig:create_vms} Creando máquinas virtuales}
\end{figure}

\begin{figure}[ht!]
  \centering
  \frame{\includegraphics[scale=0.55]{ovirt_vm_system.png}}
  \caption{\label{fig:vmsys} Opciones de sistema}
\end{figure}

\begin{figure}[ht!]
  \centering
  \frame{\includegraphics[scale=0.65]{ovirt_vms_ha.png}}
  \caption{\label{fig:vmha} Opciones de alta disponibilidad}
\end{figure}

\begin{figure}[ht!]
  \centering
  \frame{\includegraphics[scale=0.65]{ovirt_vm_host.png}}
  \caption{\label{fig:vmhost} Opciones de host}
\end{figure}

\begin{figure}[ht!]
  \centering
  \frame{\includegraphics[scale=0.65]{ovirt_vm_boot.png}}
  \caption{\label{fig:vm_boot} Arranque}
\end{figure}

\clearpage
\subsection{oVirt Guest Agent}
\label{subsec:guestagent}

Una vez instaladas las máquinas, debemos instalar el oVirt Guest Agent. Ésta herramienta permite al sistema operativo virtualizado comunicarse con el nodo maestro, ofreciéndole información tal como la dirección IP, el \emph{FQDN} y estadísticas de uso de recursos. El proceso varía dependiendo del sistema operativo, pero a modo de ejemplo, el proceso en Ubuntu es el siguiente:

\begin{enumerate}
\item Editar el fichero \texttt{/etc/apt/sources.list.d/ovirt-guest-agent.list} añadiendo el siguiente contenido:

    \begin{TMterminal}{}{}{/etc/apt/sources.list.d/ovirt-guest-agent.list}
        deb http://download.opensuse.org/repositories/home:/evilissimo:/ubuntu:/14.04/xUbuntu_14.04/ /
    \end{TMterminal}

\item Añadir la \emph{Release key}:

    \begin{TMterminal}{}{}{Añadiendo la release key}
        wget http://download.opensuse.org/repositories/home:/evilissimo:/ubuntu:/14.04/xUbuntu_14.04//Release.key
        sudo apt-key add - < Release.key  
    \end{TMterminal}

\item Y finalmente, actualizamos las listas de software e instalamos el paquete:

    \begin{TMterminal}{}{}{Instalando}
        sudo apt-get update && sudo apt-get install ovirt-guest-agent
    \end{TMterminal}
\end{enumerate}

Una vez esté instalado, el servicio se autoejecutará en cada arranque de la máquina virtual. Si ahora vamos a la interfaz web, deberíamos ver la información de la siguiente manera:

\begin{figure}[ht!]
  \centering
  \hspace*{-1cm}
  \includegraphics[scale=0.40]{ovirt-vms.png}
  \caption{\label{fig:listado_vms} Lista de máquinas virtuales}
\end{figure}












% \chapter{Conclusiones}
\label{ch:conclusiones}

\section{Objetivos cumplidos}
\label{sec:objetivos}

Durante el transcurso del trabajo, puedo considerar los siguientes objetivos como completados:

\begin{itemize}
    \item Se ha estudiado el uso de la virtualización como herramienta para asegurar servicios.
    \item Se ha estudiado el uso de oVirt como plataforma de gestión del \emph{datacenter}
    \item Assegurar els servidors de l'IES La Bastida a través de la virtualització
    \item Se ha probado la migración de máquinas virtuales, aunque con resultados variados.
    \item Estudiados los mecanismos de copias de seguridad en oVirt.
\end{itemize}

Considerando como el único objetivo que no ha sido completado satisfactoriamente de forma regular es la migración de máquinas virtuales, considero el proyecto un éxito, si menos tomándolo como una prueba de concepto de lo que se puede hacer con oVirt.

Resulta una lástima que hayan cosas que no se hayan podido completar satisfactoriamente, o averiguar por qué fallaban, como en el caso de la migración de máquinas virtuales, pero desde bastante pronto en el proyecto decidí que éste iba a ser más una prueba de concepto que un proyecto a ser implementado tal y como saliera de éste crédito.

\section{Análisis de la planificación y metodología}
\label{sec:plan_metod}

Acercándonos al final del proyecto, considero que la metodología que se decidió seguir este año, realmente no difiere en mucho a como se había hecho en años anteriores. El seguimiento del proyecto fue casi inexistente, cosa que a mí personalmente me llevó a una dinámica que no dista demasiado de cualquier otro proyecto. Por supuesto que adherise rigurosamente al plan de trabajo me correspondía a mí, pero se hubiera agradecido una mayor supervisión, sobretodo en materia de contenido del documento.

\section{Planes de futuro para el proyecto}
\label{sec:futuro}

Una vez concluido el curso, continuaré trabajando en el proyecto, y experimentando con oVirt y otras herramientas, pues el objetivo era encontrar la manera de implementar un sistema alternativo en el IES La Bastida.

Uno de los puntos que más ganas tengo de investigar es el uso de ZFS, recientemente incorporado a Ubuntu 16.04, como sistema de ficheros detrás del almacenamiento en red. En la comparativa de ZFS, btrfs, XFS, ext4 y LVM/KVM realizada por Gionatan Danti \cite{danticomparison}, ZFS sale bastante bien parada de la comparativa con los otros sistemas de ficheros, y ofrece también una serie de características adicionales, como la caché ARC, soporte nativo de \emph{snapshots}, \emph{striping} dinámico, deduplicación de datos, y más.

 Existe también otro punto importante que no he podido tratar en el proyecto por falta de tiempo, el uso de soluciones basadas en \textbf{contenedores}, como Proxmox VE. 

%% \section{Opiniones personales}
%% \label{sec:personal}



% \chapter{Anexo}
\label{ch:anexo}

\section{Instalando CentOS 7}
\label{sec:instalando_centos}

La instalación de CentOS se basa en el instalador Anaconda, que ya viene siendo usado en Red Hat Enterprise Linux y Fedora. Difiere de otros instaladores, en que en lugar de ser lineal, presenta al usuario una \emph{landing screen}, en la cual se encuentran las diferentes preguntas que típicamente se realizan durante una instalación: El particionado de los discos, idioma, distribución de teclado, paquetes a instalar\ldots La siguiente captura muestra dicha pantalla:

\begin{figure}[ht!]
  \centering
  \includegraphics[scale=0.55]{img/anaconda1.png}
  \caption{\label{fig:label} Pantalla principal del instalador Anaconda en CentOS}
\end{figure}

Algunos de los apartados son bastante sencillos y se explican ellos mimsos, así que no incluiré capturas de los mismos. 

\subsection{Localización}
\label{subsec:localizacion}

\subsubsection{Fecha y hora}
\label{subsec:fechahora}

Éste apartado es muy similar al de otros instaladores. Un mapa del planeta aparecerá, dónde podremos seleccionar nuestra ciudad. Acorde a la selección, se ajustará la hora del sistema. Es posible también desactivar NTS, y introducir tanto la fecha como la hora de forma manual.

\subsubsection{Teclado}
\label{subsec:teclado}

Bastante simple también. Aparecerán dos cuadros, uno a la izquierda y otro a la derecha. En la parte inferior del panel izquierdo habrá una lista de botones, que corresponden a Añadir, Quitar, Mover arriba y Mover abajo. Cuando seleccionemos Añadir, aparecerá una lista de idiomas, y las distribuciones de cada idioma una vez se haya seleccionado uno.

En el derecho, simplemente, se podrá probar la distribución de teclado seleccionada.

\subsubsection{Soporte de idiomas}
\label{subsec:soportedeidiomas}

La pantalla de soporte de idiomas es casi exactamente la misma que la primera que encontramos nada más arrancar el instalador, con la única diferencia que en lugar de escoger un único idioma, podemos ir marcando varios, para que luego se instale soporte para todos aquellos idiomas marcados.

\subsection{Discos}
\label{subsec:discos}

Éste es el único apartado \emph{obligatorio} para completar la instalación; todos los demás apartados pueden dejarse con los valores por defecto. Una vez accedemos al apartado, se nos presenta la siguiente pantalla:

\begin{figure}[ht!]
  \centering
  \includegraphics[scale=0.55]{centos-disk1.png}
  \caption{\label{fig:diskselect} Selección de discos}
\end{figure}

\begin{figure}[ht!]
  \centering
  \includegraphics[scale=0.55]{centos-disk2.png}
  \caption{\label{fig:add_mpoint} Añadiendo punto de montaje}
\end{figure}

\begin{figure}[ht!]
  \centering
  \includegraphics[scale=0.55]{centos-disk3.png}
  \caption{\label{fig:edit_mpoint} Editando punto de montaje}
\end{figure}

\clearpage

\subsection{Continuando la instalación}
\label{subsec:continuando}

Una vez realizado el particionamiento de los discos, la instalación se iniciará, y deberemos editar la contraseña del usuario root, y crear un usuario adicional. 


\section{Trabajando remotamente}
\label{sec:remote}

Dadas las circunstancias del proyecto (No dispongo del hardware físico en el Puig), y que ahora sólo voy aproximadamente ocho horas a la semana a la Bastida, he tenido que buscarme la vida para poder trabajar con las máquinas virtuales en el Puig.

La solución que detallaré ahora a continuación no es muy complicada, pero todo gira en torno a que La Bastida tenga conexión a Internet. Si no puedo llegar hasta su proxy, no puedo trabajar. Una vez dicho esto, lo primero que tengo que hacer para poder acceder a la máquina de la red local de la Bastida donde está instalada la interfaz web del motor, es convertir a mi portátil en un proxy SOCKS.\@Ésto lo puedo hacer mediante el siguiente túnel SSH:

\begin{TMterminal}{}{}{Creando el túnel}
  ssh -D 10080 enrique@ieslabastida.xtec.cat -p 2209
\end{TMterminal}

Mientras la sesión SSH siga activa, mi portátil estará actuando como un proxy SOCKS.\@Si configuro Firefox con dicho proxy como en la siguiente captura: 

\begin{figure}[ht!]
  \centering
  \includegraphics[scale=0.55]{proxyconf.png}
  \caption{\label{fig:Proxy} Configuración del proxy en Firefox}
\end{figure}

Todo el tráfico pasará por el puerto 10080 de mi portátil, que no es ni más ni menos un túnel SSH al Proxy de la Bastida, que atenderá la petición, y me permitirá conectarme al portal de administración.

\subsection{Consola remota mediante el proxy SOCKS}
\label{subsec:consola}

oVirt permite utilizar un cliente de consola remota para conectarse a las máquinas virtuales, y trabajar con el escritorio remoto. Ésto se hace mediante ficheros con extensión $.$vv que permiten a un programa como remote-viewer conectarse. El único problema, es que éste programa no incluye configuración de red, ni la distribución que utilizo (Xubuntu 15.10) incluye configuración global del proxy. La solución a éste problema radica entonces en la librería \textbf{tsocks}, que intercepta todo el tráfico saliente, y lo re-envía a un servidor SOCKS.\@Está pensado para aplicaciones que por defecto no tratan bien con servidores proxy SOCKS.

La siguiente figura muestra la breve configuración (sólo dos líneas) necesarias para hacer que tsocks mande el tráfico por el mismo proxy SOCKS que utilizamos antes:

\begin{TMterminal}{}{}{Configurando tsocks}
  server      = 127.0.0.1
  server_port = 10080
\end{TMterminal}

Una vez editado el fichero de configuración /etc/tsocks, ya podemos ejecutar cualquier programa tal que \texttt{tsocks} \emph{programa}, y todo el tráfico de red que éste genere irá a pasar al servidor SOCKS.









\nocite{*}
\clearpage
\glsaddall
\printglossary[nonumberlist]
\printbibliography[heading=bibintoc]

\end{document}
